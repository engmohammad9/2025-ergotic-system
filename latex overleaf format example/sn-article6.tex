%% 点击左上角菜单,将编译器选择为XeLaTeX
%% Нажмите на меню в левом верхнем углу и выберите компилятор XeLaTeX

\documentclass{article}
\usepackage[a4paper, margin=1.5cm]{geometry}
\usepackage[russian,english]{babel}
\usepackage{fontspec}
\usepackage{titlesec}
\usepackage{graphicx}
\setmainfont{Times New Roman}
\usepackage{url}
\usepackage{hyperref}
\hypersetup{colorlinks=true, citecolor=blue}
\usepackage{float}
\usepackage{ctex}
\usepackage{paracol}
\makeatletter
\providecommand{\pcsync}{\par\vspace{\baselineskip}}
\makeatother
\usepackage[backend=biber, style=ieee]{biblatex}
\usepackage{balance}

\titleformat{\section}{\normalfont\large\bfseries}{\thesection}{1em}{}
% \title{Отчёт}
\title{Отчёт  \ \large 报告}
\author{Исса Мохаммад \ 伊萨·穆罕默德}
\date{13 мая 2025 г. \ 2025年5月13日}

\begin{document}
\maketitle

\begin{paracol}{2}
\setcolumnwidth{0.47\textwidth,0.47\textwidth}
\setlength{\columnsep}{20pt}
\setlength{\emergencystretch}{3em}

\section{Введение}
\pcsync
Электрокардиография (ЭКГ) — это широко используемый метод диагностики сердечных заболеваний. Автоматическая классификация ЭКГ-сигналов может значительно повысить эффективность клинической диагностики. В рамках данной работы мы исследуем характеристики ЭКГ-сигналов и применяем методы глубинного обучения, в частности сети с долгой краткосрочной памятью (LSTM) и механизмом внимания, для классификации сердечных сокращений в соответствии со стандартом, рекомендованным AAMI.

Модель обучается и тестируется на базе MIT-BIH Arrhythmia Database. Этот набор данных включает 48 получасовых фрагментов двухканальных амбулаторных ЭКГ-записей. Аннотации сердечных сокращений приведены к пяти основным категориям согласно стандарту AAMI:

\begin{itemize}
    \item \textbf{N:} Нормальные сокращения
    \item \textbf{S:} Суправентрикулярные экстрасистолы
    \item \textbf{V:} Желудочковые экстрасистолы
    \item \textbf{F:} Слияние нормальных и желудочковых сокращений
    \item \textbf{Q:} Неизвестные или прочие
\end{itemize}

\switchcolumn

\section{引言}
\pcsync
心电图(ECG)是一种广泛用于诊断心脏疾病的技术。自动化的ECG信号分类可以提高临床诊断的效率。本研究旨在分析ECG信号特征,并应用深度学习方法,特别是带注意力机制的长短期记忆网络(LSTM),依据AAMI推荐标准对心搏进行分类。

该模型在MIT-BIH心律失常数据库上进行训练和测试。该数据库包含48段每段30分钟的双通道动态ECG记录。根据AAMI标准,心搏注释被映射为五个主要类别:

\begin{itemize}
    \item \textbf{N:} 正常心搏
    \item \textbf{S:} 室上性异位心搏
    \item \textbf{V:} 室性异位心搏
    \item \textbf{F:} 正常与室性心搏融合
    \item \textbf{Q:} 未知或其他
\end{itemize}

\switchcolumn*


\section{Предобработка данных}
\pcsync
\begin{itemize}
    \item \textbf{Выделение сердечных сокращений:} ЭКГ-сигналы сегментируются по сокращениям с использованием аннотаций R-пиков из PhysioNet.
    \item \textbf{Длина сигнала:} Каждое сокращение дополняется или обрезается до 300 отсчетов.
    \item \textbf{Преобразование меток:} Символы сокращений преобразуются в классы AAMI.
    \item \textbf{Нормализация:} Все сегменты нормализуются с использованием z-преобразования.
\end{itemize}

Датасет является сильно несбалансированным: класс ``N'' составляет более 80\% всех сокращений.

\pcsync
Примечание: Из-за ограниченных ресурсов моего компьютера я использовал только записи с 100 по 180.

\switchcolumn

\section{数据预处理}
\pcsync
\begin{itemize}
    \item \textbf{心搏提取:} 使用PhysioNet中的R峰注释将ECG信号按心搏分段。
    \item \textbf{信号长度:} 每个心搏被填充或截断为300个采样点。
    \item \textbf{标签映射:} 心搏符号被映射为AAMI分类标准中的类别。
    \item \textbf{归一化:} 所有心搏段使用z-score标准化。
\end{itemize}

该数据集严重不平衡,其中“N”类占所有心搏的80%以上。
\pcsync
备注: 由于我的计算机资源有限,我仅使用了第100到第180号的记录。

\switchcolumn*



\section{Архитектура модели}
\pcsync
Нейронная сеть включает в себя следующие компоненты:
\begin{itemize}
    \item LSTM-слой с 64 единицами
    \item Слой мягкого внимания (Dense + Multiply)
    \item Второй LSTM-слой с 32 единицами
    \item Выходной полносвязный слой с softmax-активацией на 5 классов
\end{itemize}

Общее количество обучаемых параметров: 33,637 \\
Функция потерь: категориальная кросс-энтропия \\
Оптимизатор: Adam

\begin{figure}[H]
    \centering
    \includegraphics[width=0.75\linewidth]{1.png}
    \caption{Резюме модели}
    \label{fig:enter-label}
\end{figure}


\begin{figure}[H]
    \centering
    \includegraphics[width=0.75\linewidth]{2.png}
    \caption{Распределение классов}
    \label{fig:enter-label}
\end{figure}



\switchcolumn

\section{模型结构}
\pcsync
神经网络由以下部分组成:
\begin{itemize}
    \item 一个包含64个单元的LSTM层
    \item 一个软注意力层(Dense + Multiply)
    \item 一个包含32个单元的第二LSTM层
    \item 一个具有5个节点的Dense softmax输出层
\end{itemize}

总可训练参数:33,637 \\
损失函数:分类交叉熵(Categorical Crossentropy)\\
优化器:Adam

\begin{figure}[H]
    \centering
    \includegraphics[width=0.75\linewidth]{1.png}
    \caption{模型摘要}
    \label{fig:enter-label}
\end{figure}

\begin{figure}[H]
    \centering
    \includegraphics[width=0.75\linewidth]{2.png}
    \caption{班级分布}
    \label{fig:enter-label}
\end{figure}

\switchcolumn*





\section{Результаты обучения}
\pcsync
Модель обучалась в течение 10 эпох на 80\% выборки. Во время обучения использовалось 20\% данных для валидации. Были применены веса классов для балансировки тренировочного процесса.

Несмотря на балансировку, точность модели остаётся нестабильной и низкой из-за несбалансированности классов и недообучения модели.

\begin{figure}[H]
    \centering
    \includegraphics[width=0.75\linewidth]{3.png}
    \caption{Обучение}
    \label{fig:enter-label}
\end{figure}

\switchcolumn

\section{训练结果}
\pcsync
该模型使用80\%的数据进行了10轮训练。在训练过程中使用了20\%的数据作为验证集。为了解决类别不平衡问题,训练中应用了类别权重。

尽管进行了类别平衡处理,但由于类别严重不均衡和模型欠拟合,准确率仍然不稳定且偏低。


\begin{figure}[H]
    \centering
    \includegraphics[width=0.75\linewidth]{3.png}
    \caption{训练}
    \label{fig:enter-label}
\end{figure}


\switchcolumn*








\section{Оценка качества модели}
\pcsync
\textbf{Матрица ошибок:}

Модель путает класс N с классами Q и S. Миноритарные классы, такие как F и S, предсказываются плохо.

\begin{figure}[H]
    \centering
    \includegraphics[width=0.75\linewidth]{4.png}
    \caption{Матрица ошибок}
    \label{fig:enter-label}
\end{figure}

\begin{figure}[H]
    \centering
    \includegraphics[width=0.75\linewidth]{5.png}
    \caption{Точность и потери}
    \label{fig:enter-label}
\end{figure}

\textbf{Отчёт по классификации:}
\begin{itemize}
    \item Точность (Accuracy): 12\%
    \item Macro F1-мера: 0.10 $\rightarrow$ плохая обобщающая способность по всем классам
    \item Взвешенная F1-мера: 0.08 $\rightarrow$ отражает влияние несбалансированных данных
\end{itemize}

\begin{figure}[H]
    \centering
    \includegraphics[width=0.75\linewidth]{6.png}
    \caption{Отчёт по классификации}
    \label{fig:enter-label}
\end{figure}

\textbf{Визуализация t-SNE:}

Латентные признаки, извлечённые из предпоследнего LSTM-слоя, не показывают чёткой кластеризации по классам, что указывает на слабую разделимость признаков.

\begin{figure}[H]
    \centering
    \includegraphics[width=0.75\linewidth]{7.png}
    \caption{Визуализация t-SNE}
    \label{fig:enter-label}
\end{figure}

\switchcolumn

\section{评估结果}
\pcsync
\textbf{混淆矩阵:}

模型将N类与Q类和S类混淆。少数类(如F类和S类)的预测效果较差。

\begin{figure}[H]
    \centering
    \includegraphics[width=0.75\linewidth]{4.png}
    \caption{误差矩阵}
    \label{fig:enter-label}
\end{figure}

\begin{figure}[H]
    \centering
    \includegraphics[width=0.75\linewidth]{5.png}
    \caption{准确率和损失}
    \label{fig:enter-label}
\end{figure}

\textbf{分类报告:}
\begin{itemize}
    \item 准确率(Accuracy):12\%
    \item Macro F1分数:0.10 $\rightarrow$ 各类别的泛化能力较差
    \item 加权F1分数:0.08 $\rightarrow$ 反映了类别不平衡带来的偏差
\end{itemize}

\begin{figure}[H]
    \centering
    \includegraphics[width=0.75\linewidth]{6.png}
    \caption{分类报告}
    \label{fig:enter-label}
\end{figure}

\textbf{t-SNE可视化:}

从倒数第二个LSTM层提取的潜在特征未表现出明显的类内聚集性,说明模型难以有效区分不同类别。

\begin{figure}[H]
    \centering
    \includegraphics[width=0.75\linewidth]{7.png}
    \caption{t-SNE可视化}
    \label{fig:enter-label}
\end{figure}

\switchcolumn*



\section{Заключение}
\pcsync
Текущая реализация LSTM с механизмом мягкого внимания демонстрирует ограничения при решении задачи классификации ЭКГ-сигналов с сильным дисбалансом классов. Хотя внимание теоретически должно улучшать результаты за счёт акцента на значимых временных отрезках, доминирование класса ``N'' приводит к плохому обучению модели на редких классах.

\textbf{Рекомендации по улучшению:}
\begin{itemize}
    \item Применить увеличение данных или oversampling для редких классов
    \item Ввести dropout и регуляризацию в LSTM-слои
    \item Использовать более сложные механизмы внимания (например, Bahdanau attention)
    \item Увеличить число эпох до 30+ с применением ранней остановки (early stopping)
    \item Предобработать интервалы R-R или использовать многоканальные ЭКГ-входы
\end{itemize}

Несмотря на неидеальные результаты, предложенный рабочий процесс демонстрирует полноценный пайплайн от предобработки данных до оценки нейросетевой модели и может служить прочной основой для последующих улучшений.

\switchcolumn

\section{结论}
\pcsync
目前实现的带软注意力机制的LSTM在处理高度不平衡的ECG分类任务中表现出一定的局限性。虽然注意力机制在理论上能够通过强调关键时间步提升性能,但由于“N”类的主导地位,模型在稀有类别上的学习效果较差。

\textbf{改进建议:}
\begin{itemize}
    \item 对少数类使用数据增强或过采样
    \item 在LSTM层中引入dropout和正则化
    \item 应用更高级的注意力机制(如Bahdanau注意力)
    \item 将训练扩展至30轮以上,并采用早停策略
    \item 对R-R间期进行预处理,或使用多通道ECG输入
\end{itemize}

尽管当前结果并不理想,但该流程展示了从数据预处理到神经网络评估的完整流程,为未来的改进提供了坚实的基线。

\switchcolumn*






\section{Ссылки на литературу}
\pcsync
\renewcommand{\refname}{}
\begin{thebibliography}{9}

\bibitem{logreg1}
\textit{Набор данных для лабораторных работы и исследований}. 2025. URL: \url{https://github.com/AI-is-out-there/data2lab} (Дата обращения: 16.03.2025)


\bibitem{biomedai_lstm}
TAUforPython. \textit{Классификация ЭКГ с помощью LSTM нейросети} [Электронный ресурс]. 2023. URL: \url{https://github.com/TAUforPython/BioMedAI/blob/main/NN%20LSTM%20ECG%20classification.ipynb} (Дата обращения: 14.05.2025)

\end{thebibliography}

\vfill\null

\switchcolumn

\section{参考文献}
\pcsync
\renewcommand{\refname}{}
\begin{thebibliography}{9}

\bibitem{logreg1}
AI-is-out-there. 心脏数据集 [EB/OL]. (2025-03-16) [2025-05-06]. \url{https://github.com/AI-is-out-there/data2lab}


\bibitem{biomedai_lstm}
TAUforPython. LSTM 神经网络用于ECG分类 [EB/OL]. (2023) [2025-05-14]. \url{https://github.com/TAUforPython/BioMedAI/blob/main/NN%20LSTM%20ECG%20classification.ipynb}

\end{thebibliography}

\vfill\null

\switchcolumn*








\end{paracol}
\balance
\end{document}
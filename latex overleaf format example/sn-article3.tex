%% 点击左上角菜单,将编译器选择为XeLaTeX
%% Нажмите на меню в левом верхнем углу и выберите компилятор XeLaTeX

\documentclass{article}
\usepackage[a4paper, margin=1.5cm]{geometry}
\usepackage[russian,english]{babel}
\usepackage{fontspec}
\usepackage{titlesec}
\usepackage{graphicx}
\setmainfont{Times New Roman}
\usepackage{url}
\usepackage{hyperref}
\hypersetup{colorlinks=true, citecolor=blue}
\usepackage{float}
\usepackage{ctex}
\usepackage{paracol}
\makeatletter
\providecommand{\pcsync}{\par\vspace{\baselineskip}}
\makeatother
\usepackage[backend=biber, style=ieee]{biblatex}
\usepackage{balance}

\titleformat{\section}{\normalfont\large\bfseries}{\thesection}{1em}{}
% \title{Отчёт}
\title{Отчёт  \ \large 报告}
\author{Исса Мохаммад \ 伊萨·穆罕默德}
\date{13 мая 2025 г. \ 2025年5月13日}

\begin{document}
\maketitle

\begin{paracol}{2}
\setcolumnwidth{0.47\textwidth,0.47\textwidth}
\setlength{\columnsep}{20pt}
\setlength{\emergencystretch}{3em}

\section{Цель работы}
\pcsync
Целью данной работы является автоматическая классификация ЭКГ-записей по параметру здоровья пациента (\texttt{Healthy\_Status}) с использованием современных \texttt{AutoML}-фреймворков. Анализ выполняется на базе реальных медицинских данных, включающих временные и угловые параметры электрокардиограммы.
\switchcolumn % switch to right

\section{目标}
\pcsync 
本研究旨在利用现代的 \texttt{AutoML} 框架,根据患者的 \texttt{Healthy\_Status} 参数对心电图(ECG)记录进行自动分类。分析基于真实的医疗数据,包含心电图的时间和角度参数。

\switchcolumn* % switch back to left

\section{Метод}
\pcsync
\textbf{Выбор фреймворка.} Среди популярных AutoML-решений были рассмотрены FLAML, H2O AutoML и LightAutoML. Для решения задачи бинарной классификации был выбран фреймворк \texttt{FLAML}, поскольку он продемонстрировал наилучшие результаты при ограниченном времени обучения (60 секунд), что особенно важно в условиях ограниченных вычислительных ресурсов. FLAML автоматически подбирает оптимальную модель и её гиперпараметры, ориентируясь на метрику F1-score. В рамках эксперимента также были построены ROC-кривая и матрица ошибок для оценки качества классификации.

\switchcolumn

\section{方法}
\pcsync
\textbf{框架选择。} 本研究对主流的 AutoML 框架进行了评估,包括 FLAML、H2O AutoML 和 LightAutoML。最终选择了 \texttt{FLAML} 来解决二分类任务,因为它在仅限 60 秒训练时间的条件下表现最佳,尤其适用于计算资源有限的场景。FLAML 能够根据 F1 分数自动选择最优模型和超参数。在实验中还绘制了 ROC 曲线和混淆矩阵,以评估分类器的性能。

\switchcolumn*

\section{Обсуждение}
\pcsync
Исходный набор данных содержал более 30 признаков. После этапов очистки и отбора были оставлены ключевые переменные: \texttt{Count\_subj}, \texttt{rr\_interval}, \texttt{p\_end}, \texttt{qrs\_onset}, \texttt{qrs\_end}, \texttt{p\_axis}, \texttt{qrs\_axis}, \texttt{t\_axis}. 

В результате обучения классификатор продемонстрировал следующие показатели:

\begin{itemize}
\item Точность (Accuracy): 85\%
\item F1-мера для класса 1: 0.64
\item Площадь под ROC-кривой (AUC): 0.92
\end{itemize}

Полученные результаты свидетельствуют о том, что модель способна эффективно различать два класса, несмотря на частичное перекрытие распределений признаков.


\begin{figure}[H]
    \centering
    \includegraphics[width=0.75\linewidth]{confusion_matrix.png}
    \caption{Матрица ошибок (Confusion Matrix)}
    \label{fig:enter-label}
\end{figure}

Матрица ошибок отображает распределение предсказанных и фактических меток классов модели. Высокая концентрация значений на диагонали указывает на хорошую точность классификации.

\begin{figure}[H]
\centering
\includegraphics[width=0.75\linewidth]{classification_report.png}
\caption{Отчёт классификации}
\end{figure}

Классификационный отчёт предоставляет значения точности, полноты и F1-меры для каждого класса. Он позволяет оценить сбалансированность модели по метрикам.

\begin{figure}[H]
\centering
\includegraphics[width=0.75\linewidth]{roc_curve.png}
\caption{ROC-кривая}
\end{figure}

ROC-кривая иллюстрирует соотношение между истинно положительными и ложноположительными результатами. Площадь под кривой (AUC) равная 0.92 говорит о высокой способности модели различать классы.

\begin{figure}[H]
\centering
\includegraphics[width=0.75\linewidth]{feature_importance.png}
\caption{Важность признаков}
\end{figure}

График важности признаков демонстрирует вклад каждого входного параметра в предсказание модели. Наиболее значимые признаки оказывают наибольшее влияние на результат классификации.

\switchcolumn

\section{讨论}
\pcsync
原始数据集中包含了 30 多个特征变量。经过数据清洗和筛选后,保留了以下关键特征:Count\_subj、rr\_interval、p\_end、qrs\_onset、qrs\_end、p\_axis、qrs\_axis、t\_axis。训练得到的分类模型具有以下性能指标:

\begin{itemize}
\item 准确率(Accuracy):85\%
\item F1 分数(类别 1):0.64
\item AUC 值:0.92
\end{itemize}

这表明尽管两个类别在数据上存在重叠,模型仍能成功区分它们。

\begin{figure}[H]
    \centering
    \includegraphics[width=0.75\linewidth]{confusion_matrix.png}
    \caption{混淆矩阵}
    \label{fig:enter-label}
\end{figure}

混淆矩阵展示了模型预测结果与实际类别之间的对应关系。对角线上的高密度表示分类器具有良好的准确性。



\begin{figure}[H]
\centering
\includegraphics[width=0.75\linewidth]{classification_report.png}
\caption{分类报告}
\end{figure}

分类报告列出了每个类别的精确率、召回率和 F1 分数,用于评估模型的性能与平衡性。

\begin{figure}[H]
\centering
\includegraphics[width=0.75\linewidth]{roc_curve.png}
\caption{ROC 曲线}
\end{figure}

ROC 曲线展示了模型在不同阈值下的真阳性率与假阳性率之间的权衡。AUC 值为 0.92 表明模型区分类别的能力很强。


\begin{figure}[H]
\centering
\includegraphics[width=0.75\linewidth]{feature_importance.png}
\caption{特征重要性}
\end{figure}

特征重要性图反映了各输入变量对模型预测的贡献程度。得分较高的特征对最终分类结果影响最大。

\switchcolumn*


\section{Выводы}
\pcsync
Проведённый анализ подтвердил, что AutoML-фреймворк \texttt{FLAML} способен эффективно решать задачу классификации ЭКГ-записей с высокой точностью (AUC 0.92). Использование автоматизированных подходов сокращает время настройки модели и повышает воспроизводимость результатов.

\switchcolumn

\section{结论}
\pcsync
实验结果表明,\texttt{FLAML} 框架可以有效地完成心电图分类任务,AUC 值高达 0.92,表明模型具有很强的判别能力。自动机器学习显著简化了建模流程,提高了结果的稳定性与效率。
\switchcolumn*
\newpage



\section{Ссылки на литературу}
\pcsync
\renewcommand{\refname}{}
\begin{thebibliography}{9}
\bibitem{logreg1}
\textit{Набор данных для лабораторных работы и исследований}. 2025. URL: \url{https://github.com/AI-is-out-there/data2lab} (Дата обращения: 16.03.2025)

\bibitem{logreg2}
\textit{FLAML: A Fast Library for AutoML}. 2025. URL: \url{https://github.com/microsoft/FLAML} (Дата обращения: 06.05.2025)
\end{thebibliography}

\vfill\null  % <-- This pushes content down to match Chinese side, optional

\switchcolumn

\newpage
\section{参考文献}
\pcsync
\renewcommand{\refname}{}
\begin{thebibliography}{9}
\bibitem{logreg1}
AI-is-out-there. 心脏数据集 [EB/OL]. (2025-03-16) [2025-05-06]. \url{https://github.com/AI-is-out-there/data2lab}

\bibitem{logreg2}
Microsoft. FLAML 自动机器学习库 [EB/OL]. (2025-05-06) [2025-05-06]. \url{https://github.com/microsoft/FLAML}
\end{thebibliography}

\vfill\null  % Optional: ensures vertical sync with Russian side

\switchcolumn*


\end{paracol}
\balance
\end{document}